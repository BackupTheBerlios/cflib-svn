This library started as a Beginner Project in \char`\"{}C Library Building\char`\"{} following practical needs arising from scientific/technical modeling projects.

Most of the library code has been created in 1994/95 on an Atari ST4/16 MHz under TOS 2.05/MiNT 1.12 with gcc 2.5.8 and the Mintlibs Patchlevel 46. The C coding is probably not the best possible ;-) but the source as well as the executable and allocated memory structures are very compact, simple to modify and still fine for many applications that need a stable, portable and small configuration interface.

\begin{Desc}
\item['Mission Statement' from 1994/95 README file:]CFLIB is meant to be a flexible, sound and easy to use tool for C programmers. It provides a set of functions for a standard method for feeding a C program with all the (external) information it needs to perform as desired with a minimum of expense for both the programmer and the user: arguments, commands, program input, configurable and/or installation dependent features, system settings and a lot more can be passed to the program through different interfaces: command line, environment, configuration files or sections within them, interactive input and last but not least a built-in default. So it should be a fast and easy task for both the programmer and the user to build and change a configuration. The library will always take more space and perform slower than code that is written and optimized for a specific requirement, but this will only be a noticeable disadvantage in some cases. The library also provides some special features such as file search, time and date handling, generating simple, text template driven reports, etc. It is suitable for creating a comfortable and/or individual user interface for existing programs that don't have one. Some basic ideas for the library came from looking at the {\em termcap\/} library for easy and portable terminal I/O. \par
 \end{Desc}
