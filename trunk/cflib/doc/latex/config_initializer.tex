The {\bf \char`\"{}Configuration Initializer\char`\"{}} is the major interface between the user program and CFLIB.

It is an Array of \hyperlink{struct_c_o_n_f_i_g}{CONFIG} items, just as the CFLIB Configuration Database itself.

Every parameter is characterized by:

\begin{itemize}
\item \hyperlink{config_initializer_parameter_name}{Parameter Name} (\hyperlink{struct_c_o_n_f_i_g_196d63e311e3d1aa79fe4f16f20ce2a5}{CONFIG::name})\end{itemize}


\begin{itemize}
\item \hyperlink{config_initializer_parameter_default}{Parameter Default Value} (\hyperlink{struct_c_o_n_f_i_g_b8cae13203b07fabaa40e407fa8dbdfb}{CONFIG::inhalt})\end{itemize}


\begin{itemize}
\item \hyperlink{config_initializer_parameter_option}{Commandline Option for Parameter} (\hyperlink{struct_c_o_n_f_i_g_60563bd93e85c1ddb8291a2d27a9c472}{CONFIG::option})\end{itemize}


\begin{itemize}
\item \hyperlink{config_initializer_parameter_flagmask}{Special Option Flags Mask for Parameter} (\hyperlink{struct_c_o_n_f_i_g_b04d08abdf758c0400caaded716f4089}{CONFIG::flag})\end{itemize}


An example initializer may look like this: 

\begin{Code}\begin{verbatim} const CONFIG initializer[] = {
       // name        default        option  flag
     { "SPI_PROFILE", "~/.spi.cnf",  'c',    CF_SETFILE|CF_SET_ENV,     },
     { "SPI_SYSCONF", "/etc/spi.cnf",' ',    CF_SYS_SETFILE|CF_NOSAVE,  },
     { "SPI_DEFS",    "loop-0.A",    'f',    CF_SECTION,                },
     { "SPI_ID",      "",            ' ',    CF_SET_PRIV,               },
     { "verbosity",   "1",           'v',    CF_INT|CF_CONCAT|CF_NOSAVE,},
     { "label_items", CF_FLAG_ON,    'l',    CF_FLAG,                   },
     { "label_date",  NULL,          ' ',    CF_DATE|CF_SET_PUT,        },
     { "label_logo",  NULL,          'L',    0,                         },
     { "output_file", NULL,          'o',    CF_CONCAT|CF_QUERY,        },
     { "TERM",        "0.2",         'T',    CF_STR|CF_IGN_ENV,         },
     { "SCALE",       "1",           'S',    CF_REAL|CF_IGN_ENV,        },
     { "LOOPMOD",     "random",      '1',    CF_NO_OPT_ARG,             },
     { "CYCLES",      "2009",        '2',    CF_INT|CF_NO_OPT_ARG,      },
     { "CF_DUMPVERB", CFD_COLHEAD,   ' ',    CF_INT|CF_NOSAVE|CF_LAST,  },
 };
\end{verbatim}
\end{Code}



The Configuration Initializer controls the behavior and actions of all {\bf cf$\ast$}() functions, starting with the control of \hyperlink{config_levels_initialization_process}{Parsing Levels in the Initialization Process}.\hypertarget{config_initializer_parameter_name}{}\subsection{Parameter Name}\label{config_initializer_parameter_name}
Every entry in the initializer must have a name, that is a non-empty string, which is searched for in the environment and in the configuration files and which is used as an argument to the inquiry functions. Parameters that have a name beginning with {\bf \char`\"{}CF\_\-\char`\"{}} may have special meaning, see \hyperlink{parameter_types_cflib_parameters}{CFLIB Parameters}.\hypertarget{config_initializer_parameter_default}{}\subsection{Parameter Default Value}\label{config_initializer_parameter_default}
As the default content of an entry you can specify a {\tt NULL} pointer or a string, which may be empty. If \hyperlink{group__special__options__mask_g1d1f1d1b6eac6b5d9970102318ab2667}{CF\_\-FLAG} is set for that entry you should use the \hyperlink{group__cflib__core_g355c714f2912ac336b8b03468c978d8c}{CF\_\-FLAG\_\-ON} or \hyperlink{group__cflib__core_g7010abac2c80c121772da4d9c03332ee}{CF\_\-FLAG\_\-OFF} macros.\hypertarget{config_initializer_parameter_option}{}\subsection{Commandline Option for Parameter}\label{config_initializer_parameter_option}
You can control whether and how an entry's content can be set from the Commandline by setting one of the following {\bf \char`\"{}Option Specifier\char`\"{}} Characters (and Special Option Flags where indicated):

\begin{itemize}
\item '{\em  \/}' (blank)\begin{itemize}
\item No Commandline Option for this Parameter\end{itemize}
\end{itemize}
\begin{itemize}
\item '{\em c\/}' ({\em c\/} = any \char`\"{}normal\char`\"{} character)\begin{itemize}
\item Looks for '-c $<${\em string$>$'\/} on the Commandline\item Looks for '-c$<${\em string$>$'\/} on the Commandline, if Flag \hyperlink{group__special__options__mask_g9e526dae88bf6803772107ff283637b5}{CF\_\-CONCAT} is set in the Special Options Mask\item Looks for '-c$<${\em char$>$'\/} on the Commandline, if Flag \hyperlink{group__special__options__mask_g1d1f1d1b6eac6b5d9970102318ab2667}{CF\_\-FLAG} is set in the Special Options Mask\end{itemize}
\end{itemize}
\begin{itemize}
\item '{\em \#\/}' ({\em \#\/} = a positive number = 1, 2, 3, ...)\begin{itemize}
\item Looks for the 1st, 2nd, 3rd, ... '$<${\em string$>$'\/} that is not part of an option, if Flag \hyperlink{group__special__options__mask_g97d17b070dac10c14d3930c58bfba76f}{CF\_\-NO\_\-OPT\_\-ARG} is set in the Special Options Mask\end{itemize}
\end{itemize}
\hypertarget{config_initializer_parameter_flagmask}{}\subsection{Special Option Flags Mask for Parameter}\label{config_initializer_parameter_flagmask}
Most of the magic happens here! ;-) The {\bf \char`\"{}Special Options Mask\char`\"{}} is a Bitmask of type \hyperlink{group__special__options__mask_g4854f1596d5c6e0604a478fa9a2e23f0}{CFFLAGTYP} in Configuration Entry Structure Member \hyperlink{struct_c_o_n_f_i_g_b04d08abdf758c0400caaded716f4089}{CONFIG::flag} containing Type, Instruction and Information Flags for a Parameter.  See \hyperlink{group__special__options__mask}{Special Options Mask} for details.

\begin{Desc}
\item[See also:]\hyperlink{parameter_types}{Types of Parameters} 

\hyperlink{config_levels}{Configuration Parsing Levels and Source/Origin Options}\end{Desc}
\begin{Desc}
\item[Attention:]The last entry in the initializer list must have the \hyperlink{group__special__options__mask_ga4d82cea91ede4aee78594417894e368}{CF\_\-LAST} Flag set! \end{Desc}
